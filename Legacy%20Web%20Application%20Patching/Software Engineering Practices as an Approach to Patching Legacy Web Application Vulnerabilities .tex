\documentclass[conference]{IEEEtran}
\IEEEoverridecommandlockouts
% The preceding line is only needed to identify funding in the first footnote. If that is unneeded, please comment it out.
\usepackage{cite}
\usepackage{amsmath,amssymb,amsfonts}
\usepackage{algorithmic}
\usepackage{graphicx}
\usepackage{textcomp}
\usepackage{xcolor}
\def\BibTeX{{\rm B\kern-.05em{\sc i\kern-.025em b}\kern-.08em
    T\kern-.1667em\lower.7ex\hbox{E}\kern-.125emX}}
\begin{document}

\title{Software Engineering Practices as an Approach to Patching Legacy Web Application Vulnerabilities\\
%{\footnotesize \textsuperscript{*}Note: Sub-titles are not captured in Xplore and
%should not be used}
\thanks{Identify applicable funding agency here. If none, delete this.}
}

\author{\IEEEauthorblockN{1\textsuperscript{st} Nick Stone}
\IEEEauthorblockA{\textit{Xavier University} \\
Cincinnati Ohio, USA \\
Stonen2@Xavier.edu}
\and
\IEEEauthorblockN{2\textsuperscript{nd} Deep Ramanayake}
\IEEEauthorblockA{\textit{Xavier University} \\
Cincinnati Ohio, USA \\
Ramanayaked@Xavier.edu}

}

\maketitle

\begin{abstract}
Maintaining software can be challenging especially in light of the fact that everyday new vulnerabilities are found throughout the software world. The problem becomes even more complex when a software engineer has the incredible task of patching a vulnerability. The patch that is developed needs to solve the problem while not impacting the functionality of the software product. Additionally, these software patches need to be carefully considered such that the patch does not cause more vulnerabilities as well as the patch needs to ensure that the vulnerability is solved, this meaning that the vulnerability will not come back if a software update is pushed. The API presented aims to bring together software engineering and cybersecurity to evaluate software engineering approaches to patching legacy web application vulnerabilities.  
\end{abstract}

\begin{IEEEkeywords}
API, Legacy Application, Software Patch, Vulnerability
\end{IEEEkeywords}

\section{Introduction}
Placement text to start the introduction

\section{Vulnerabilities in Question}

\subsection{Javascript Injection}

The IEEEtran class file is used to format your paper and style the text. All margins, 
column widths, line spaces, and text fonts are prescribed; please do not 
alter them. You may note peculiarities. For example, the head margin
measures proportionately more than is customary. This measurement 
and others are deliberate, using specifications that anticipate your paper 
as one part of the entire proceedings, and not as an independent document. 
Please do not revise any of the current designations.
\subsection{Cross Site Scripting}


\subsection{SQL Injection}



\section{Javascript Injection Protection}

\section{Cross Site Scripting protection}

\section{SQL Injection Protection}
Before you begin to format your paper, first write and save the content as a 
separate text file. Complete all content and organizational editing before 
formatting. Please note sections \ref{AA}--\ref{SCM} below for more information on 
proofreading, spelling and grammar.

Keep your text and graphic files separate until after the text has been 
formatted and styled. Do not number text heads---{\LaTeX} will do that 
for you.

\subsection{Making the SQL Injection Work For Us}\label{AA}
Define abbreviations and acronyms the first time they are used in the text, 
even after they have been defined in the abstract. Abbreviations such as 
IEEE, SI, MKS, CGS, ac, dc, and rms do not have to be defined. Do not use 
abbreviations in the title or heads unless they are unavoidable.

\seciton{The Importance of Error Handling}

\section{Difference in Application Paradigm}
\subsection{Object Oriented}
\subsection{Functional}

\section{Web Application Architecture}

\section{The API}

\subsection{Java}
\subsubsection{Javascript Injection}
\subsubsection{Cross Site Scripting}
\subsubsection{SQL Injection}

\subsection{PHP}
\subsubsection{Javascript Injection}
\subsubsection{Cross Site Scripting}
\subsubsection{SQL Injection}


\section{Can We Trust Cookies?} 


\section{Evaluation}

\section{Future Work}

\section*{Acknowledgment}

Thank you to Xavier University for allowing me to perform research on this unusual and uncanny topic. 

\section*{References}

Please number citations consecutively within brackets \cite{IEEEhowto:IEEEtranpage}. The 
sentence punctuation follows the bracket \cite{b2}. Refer simply to the reference 
number, as in \cite{b3}---do not use ``Ref. \cite{b3}'' or ``reference \cite{b3}'' except at 
the beginning of a sentence: ``Reference \cite{b3} was the first $\ldots$''

Number footnotes separately in superscripts. Place the actual footnote at 
the bottom of the column in which it was cited. Do not put footnotes in the 
abstract or reference list. Use letters for table footnotes.

Unless there are six authors or more give all authors' names; do not use 
``et al.''. Papers that have not been published, even if they have been 
submitted for publication, should be cited as ``unpublished'' \cite{b4}. Papers 
that have been accepted for publication should be cited as ``in press'' \cite{b5}. 
Capitalize only the first word in a paper title, except for proper nouns and 
element symbols.

For papers published in translation journals, please give the English 
citation first, followed by the original foreign-language citation \cite{b6}.

\bibliographystyle{./bibliography/IEEEtran}
\bibliography{./bibliography/IEEEabrv,./bibliography/IEEEexample}

\vspace{12pt}
\color{red}
IEEE conference templates contain guidance text for composing and formatting conference papers. Please ensure that all template text is removed from your conference paper prior to submission to the conference. Failure to remove the template text from your paper may result in your paper not being published.

\end{document}
